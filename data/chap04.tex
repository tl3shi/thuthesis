
%%% Local Variables: 
%%% mode: latex
%%% TeX-master: t
%%% End: 

\chapter{总结与展望}
\label{cha:summery:futurework}
\section{总结}
\label{sec:summery}

凸包围体在计算机图形学、计算机动画等领域中处于重要位置,常常作为原始模型的近似被广泛应用于光线跟踪、碰撞检测等算法中。常见的有沿坐标轴方向的包围盒,包围球、凸包等,包围体的紧致程度直接影响着相应算法的效率,对于一般不规则形状的模型,包围盒往往不够紧致而凸包很紧致但因其含有过多的面片数量导致算法复杂性增加。
本文提出了一种构造凸包围~$k$~面体($k$-CBP)的方法,该方法扫描输入模型的点集,然后构造一个粗糙的近似内凸包,利用这近似内凸包面片的法向通过~$k$-means~聚类算法生成指定~$k$~个构造凸包围多面体的法向;
进而利用这些法向搜索原始模型点集中的切点构成凸包围多面体的截面;最后由这些截面求交构成凸包围多面体。在搜索截面的过程中,需要根据每个法向多次扫描点集,本文提出了两种方案进行并行加速,一种是基于~OpenGL~着色语言进行并行加速,
该方法在点集较小的模型中利用深度缓存算法根据~OpenGL~渲染机制中深度裁剪自动提取切点,在点集较大的模型中利用乒乓技术能够有效对多次迭代数据进行缓冲进而能够较快获得切点;另一种方案是基于现代显卡较为通用的~CUDA~并行计算架构平台
进行并行规约加速。实验结果证明本文提出的方法能够有效提高构造凸包围多面体的构造速度,与现有算法相比平均能够加速~3 $\sim$ 8~倍。本文提出的~$k$-CBP~可根据不同应用场景选择不同参数~$k$~得到不同紧致程度的凸包围多面体。

本文提出的构造~$k$-CBP~方法较~$k$-DOP~而言能够得到更加紧致的凸包围多面体,对于一般不规则模型在~$20 \leq k \leq 40 $~时都能达到约~90\%~的紧致程度,较~$k$-DOP~能够提高约~10\% $\sim$ 40\%~的紧致程度。更加紧致的凸包围多面体能够使碰撞检测算法更有效率。本文提出一种基于~$k$-CBP~过滤的碰撞检测算法,该算法在包围盒相交后进行~$k$-CBP~的相交检测,若两个模型的~$k$-CBP~相交后再进行模型的相交测试。在进行模型的~$k$-CBP~相交测试中,本文利用了两种算法进行,一种是构造~$k$-CBP~的~AABB~树,
将~$k$-CBP~看作是传统的网格模型进行相交检测判断,另外一种是利用~GJK~算法对~$k$-CBP~进行相交检测判断,实验结果表明当模型点集较小时,基于~AABB~树的方案较优,而当模型点较大或运动场景的碰撞检测算法中,基于~GJK~算法较优。总体看来,本文
提出的基于~$k$-CBP~的碰撞检测算法在应用与静态或运动场景的碰撞检测算法中,基于~AABB~树方法或基于~GJK~算法能够保持同步,基于~$k$-DOP~树算法在初始化步骤中需要计算层次结构~$k$-DOP~耗时太久,因此本文算法在静态碰撞检测环境中都优于基于~$k$-DOP~的方法;
在运动场景的碰撞检测算法中,本文算法在~$k$-CBP~相交后,模型进一步相交检测须依赖于基于~AABB~树的碰撞检测算法,在真实模型相交的情况下,基于~$k$-DOP~树的算法优于本文采用基于~AABB~树的算法,但在频繁的碰撞环境中,本文更加紧致的~$k$-CBP~有较高的命中率,能够排除更多模型不相交的情况,因此节省了时间提高算法的整体效率。
从整体角度来看,本文算法在碰撞检测初始化过程效率上提高了至少8倍,而静止场景的碰撞检测过程中本文算法效率是基于~$k$-DOP~树的算法的~0.8 $\sim$ 3.2~倍而在运动场景中是基于~$k$-DOP~树的算法的~0.8 $\sim$ 5.6~倍。

\section{展望}
\label{sec:futurework}

凸包围多面体在计算机辅助造型设计、计算机图形学等领域里有多种应用,碰撞检测是最其中重要的应用之一。
本文提出的基于~$k$-CBP~算法在真实模型相交时需要用模型~AABB~树进行相交检测,因此算法依赖于模型~AABB~树的划分和遍历,
在未来的工作中可以考虑如何摆脱对~AABB~树的依赖,通过快速构造的~$k$-CBP~在碰撞检测过程中加快过滤过程;
或者可考虑将此~$k$-CBP~应用于近似碰撞检测算法中,用模型的多层次结构的~$k$-CBP~树替换精确的模型;
或者可考虑将此算法应用于可变形的模型连续碰撞检测,如何快速更新~$k$-CBP~也是值得研究的方向之一。
同时在未来工作可考虑如何将~$k$-CBP~应用于如机器人抓取、路径规划等其他应用领域中。
