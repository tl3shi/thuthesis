
%%% Local Variables:
%%% mode: latex
%%% TeX-master: t
%%% End:
\secretlevel{绝密} \secretyear{2100}

\ctitle{凸包围多面体生成算法及应用}
% 根据自己的情况选,不用这样复杂
\makeatletter
\ifthu@bachelor\relax\else
  \ifthu@doctor
    \cdegree{工学博士}
  \else
    \ifthu@master
      \cdegree{工学硕士}
    \fi
  \fi
\fi
\makeatother


\cdepartment[软件学院]{软件学院}
\cmajor{软件工程}
\cauthor{唐磊} 
\csupervisor{雍俊海教授}
% 如果没有副指导老师或者联合指导老师,把下面两行相应的删除即可。
% \cassosupervisor{陈文光教授}
% \ccosupervisor{某某某教授}
% 日期自动生成,如果你要自己写就改这个cdate
%\cdate{\CJKdigits{\the\year}年\CJKnumber{\the\month}月}

% 博士后部分
% \cfirstdiscipline{计算机科学与技术}
% \cseconddiscipline{系统结构}
% \postdoctordate{2009年7月——2011年7月}

\etitle{Convex Bounding Polyhedron Construction and its Application} 
% 这块比较复杂,需要分情况讨论:
% 1. 学术型硕士
%    \edegree:必须为Master of Arts或Master of Science(注意大小写)
%              “哲学、文学、历史学、法学、教育学、艺术学门类,公共管理学科
%               填写Master of Arts,其它填写Master of Science”
%    \emajor:“获得一级学科授权的学科填写一级学科名称,其它填写二级学科名称”
% 2. 专业型硕士
%    \edegree:“填写专业学位英文名称全称”
%    \emajor:“工程硕士填写工程领域,其它专业学位不填写此项”
% 3. 学术型博士
%    \edegree:Doctor of Philosophy(注意大小写)
%    \emajor:“获得一级学科授权的学科填写一级学科名称,其它填写二级学科名称”
% 4. 专业型博士
%    \edegree:“填写专业学位英文名称全称”
%    \emajor:不填写此项
\edegree{Master of Science} 
\emajor{Software Engineering} 
\eauthor{Tang Lei} 
\esupervisor{Professor Yong Junhai} 
% \eassosupervisor{Chen Wenguang} 
% 这个日期也会自动生成,你要改么?
% \edate{December, 2005}

% 定义中英文摘要和关键字
\begin{cabstract}
  
包围盒在计算机图形学和计算几何领域中应用广泛, 常用于加速几何求交、光线跟踪和碰撞
检测等多种算法. 凸包围多面体是包围盒的推广, 对于一般不规则形体, 可达到比包围盒更好的紧致
程度. 

本文提出一种快速构造给定点集的紧致凸包围多面体的方法. 该方法首先根据点集的近似凸
包, 通过$k$-means 算法生成$k$个截面法向, 然后利用GPU 沿各法向搜索切点构成截面, 最后求交构
成多面体. 实验结果表明, 与同类算法相比,
该方法能够更快地构造给定点集更紧致的凸包围多面体. 

[TODO]
并能有效加速碰撞检测算法.

%  本文的创新点主要有:
%  \begin{itemize}
%    \item 用例子来解释模板的使用方法;
%    \item 用废话来填充无关紧要的部分;
%    \item 一边学习摸索一边编写新代码。
%  \end{itemize}

\end{cabstract}

\ckeywords{凸包围体, 近似凸包, 并行计算, 碰撞检测}

\begin{eabstract} 
   [TODO] An abstract of a dissertation is a summary and extraction of research work
   and contributions. Included in an abstract should be description of research
   topic and research objective, brief introduction to methodology and research
   process, and summarization of conclusion and contributions of the
   research. An abstract should be characterized by independence and clarity and
   carry identical information with the dissertation. It should be such that the
   general idea and major contributions of the dissertation are conveyed without
   reading the dissertation. 

   An abstract should be concise and to the point. It is a misunderstanding to
   make an abstract an outline of the dissertation and words ``the first
   chapter'', ``the second chapter'' and the like should be avoided in the
   abstract.

   Key words are terms used in a dissertation for indexing, reflecting core
   information of the dissertation. An abstract may contain a maximum of 5 key
   words, with semi-colons used in between to separate one another.
\end{eabstract}

\ekeywords{Convex bounding volume, Approximate Convex hull, Parallel computing,
Collision detection}
