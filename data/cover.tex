
%%% Local Variables:
%%% mode: latex
%%% TeX-master: t
%%% End:
\secretlevel{绝密} \secretyear{2100}

\ctitle{凸包围多面体生成算法及应用}
% 根据自己的情况选,不用这样复杂
\makeatletter
\ifthu@bachelor\relax\else
  \ifthu@doctor
    \cdegree{工学博士}
  \else
    \ifthu@master
      \cdegree{工学硕士}
    \fi
  \fi
\fi
\makeatother


\cdepartment[软件学院]{软件学院}
\cmajor{软件工程}
\cauthor{唐磊} 
\csupervisor{雍俊海教授}
% 如果没有副指导老师或者联合指导老师,把下面两行相应的删除即可。
% \cassosupervisor{陈文光教授}
% \ccosupervisor{某某某教授}
% 日期自动生成,如果你要自己写就改这个cdate
%\cdate{\CJKdigits{\the\year}年\CJKnumber{\the\month}月}

% 博士后部分
% \cfirstdiscipline{计算机科学与技术}
% \cseconddiscipline{系统结构}
% \postdoctordate{2009年7月——2011年7月}

\etitle{Convex Bounding Polyhedron Construction and its Application} 
% 这块比较复杂,需要分情况讨论:
% 1. 学术型硕士
%    \edegree:必须为Master of Arts或Master of Science(注意大小写)
%              “哲学、文学、历史学、法学、教育学、艺术学门类,公共管理学科
%               填写Master of Arts,其它填写Master of Science”
%    \emajor:“获得一级学科授权的学科填写一级学科名称,其它填写二级学科名称”
% 2. 专业型硕士
%    \edegree:“填写专业学位英文名称全称”
%    \emajor:“工程硕士填写工程领域,其它专业学位不填写此项”
% 3. 学术型博士
%    \edegree:Doctor of Philosophy(注意大小写)
%    \emajor:“获得一级学科授权的学科填写一级学科名称,其它填写二级学科名称”
% 4. 专业型博士
%    \edegree:“填写专业学位英文名称全称”
%    \emajor:不填写此项
\edegree{Master of Science} 
\emajor{Software Engineering} 
\eauthor{Tang Lei} 
\esupervisor{Professor Yong Junhai} 
% \eassosupervisor{Chen Wenguang} 
% 这个日期也会自动生成,你要改么?
% \edate{December, 2005}

% 定义中英文摘要和关键字
\begin{cabstract}
  
在计算机辅助设计、计算机动画和计算机图形学等领域中,包围盒的应用十分广泛,根据其相交测试比原始模型更简单这个性质,常用于在模型之间的相关计算(如几何求交、光线跟踪或者碰撞检测等)之前进行预判剪枝,以提高整体算法的效率。
凸包围多面体作为包围盒的推广,对于一般不规则形体,可达到比包围盒更好的紧致程度,因而能够更好地进行预判剪枝以提高算法的整体效率。

本文提出了一种能够快速构造给定点集的指定~$k$~面的紧致凸包围多面体($k$ - Convex Bounding Polyhedron,简称~$k$-CBP)的方法。
该方法首先利用一个线性算法对输入点集构造一个近似内凸包,然后根据该近似凸包的面片法向通过~$k$-means~聚类算法生成构造凸包围多面体的~$k$~个截面法向,
再依次沿各个法向搜索切点构造构成凸包围多面体的截面,最后通过截面求交构成~$k$-CBP。 
在搜索截面的过程中,各个法向之间的搜索过程相互独立,可以方便地进行并行搜索,本文分别就基于~OpenGL~着色语言(GLSL)和~计算统一设备架构(CUDA)~两种平台提供了并行加速的方案。
在截面求交过程中,本文利用计算几何中的对偶映射技术加快求交过程。
实验结果表明,与同类算法相比,本文方法能够更快地构造给定点集更加紧致的凸包围多面体。

碰撞检测算法一直是计算机动画和计算机图形学领域中的研究热点,本文在层次结构包围盒树的基础上,
利用构造模型的~$k$-CBP~进行预判剪枝,提出了一种基于~$k$-CBP~的碰撞检测算法。
该方法首先构造模型的包围盒,通过包围盒相交检测的模型须通过~$k$-CBP~的相交检测才能进行真实模型的相交检测。
本文在~$k$-CBP~之间的相交测试过程中,利用了一种基于包围盒树的方法和一种基于计算凸体模型之间的最近距离的方法进行相交测试,
实验结果表明该方法在静止或运动场景的碰撞检测环境中均达到良好的剪枝效果,有助于提高碰撞检测算法的效率。


%  本文的创新点主要有:
%  \begin{itemize}
%    \item 用例子来解释模板的使用方法;
%    \item 用废话来填充无关紧要的部分;
%    \item 一边学习摸索一边编写新代码。
%  \end{itemize}

\end{cabstract}

\ckeywords{凸包围体, 近似凸包, 并行计算, 碰撞检测}

\begin{eabstract} 
Bounding box is widely applied to computer-aided design, computer animation, computer graphics and related fields. 
According to the property that bounding box is simpler than the original model, the calculations between original models can be pruned if the precomputed bounding boxes do not intersect, so that the efficiency of the algorithms, such as geometry intersection, ray tracing, collision detection and etc., can be increased.
Convex bounding polyhedron, which is a generalization of bounding box,  can approximate a model with higher tightness compared to the corresponding bounding box in usual cases, and therefore, can be used to prune more calculations.

This paper proposes a method to construct the convex bounding polyhedron made of given $k$ faces (k-Convex Bounding Polyhedron, $k$-CBP for short) from a point set.
Based on an algorithm with linear time complexity to get an approximate convex hull from the input point set, 
the algorithm generates $k$ normals by $k$-means algorithm from the normals of the approximate convex hull,  
then searches the tangent point to generate a cutting plane along each normal and at last constructs the convex bounding polyhedron by getting the intersection of the cutting planes. 
It is easy to use GPU to accelerate parallelly the searching process of the cutting planes which is independent between each other. 
This paper provides methods on both OpenGL Shading Language(GLSL) and Compute Unified Device Architecture(CUDA) platform to parallelize it. 
Technology of duality mapping in computational geometry is used to accelerate the process of getting the intersection of cutting planes. 
Experiments show that the algorithm can construct tighter convex bounding polyhedron faster compared to the related algorithms from the given point set.

Collision detection has always been a hot topic in the field of computer animation and computer graphics. 
This paper proposes a method with which the constructed $k$-CBP is used to avoid redundant operations between models if their $k$-CBPs do not intersect with each other based on the bounding volume hierarchies.
This method first constructs the bounding box of models, the models do the intersection test should first pass the pruning intersection test of bounding boxes and then the intersection test of $k$-CBPs. 
Bounding box hierarchy and a method used to calculate the minimized distance between two convex objects are used during the process of intersection test between $k$-CBPs. 
Experiments illustrate that this method can achieve good pruning effects in both static and moving environments so that it is able to speed up the process of collision detection.

\end{eabstract}

\ekeywords{Convex bounding volume, Approximate convex hull, Parallel computing,
Collision detection}
