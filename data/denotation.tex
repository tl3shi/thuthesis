\begin{denotation}

\item[BV] 包围体~(Bounding Volume)
\item[CH] 凸包~(Convex Hull)
\item[$\tau$] 包围体的紧致性 
\item[$\Delta G$]  	活化自由能~(Activation Free Energy)
\item [$\chi$] 传输系数~(Transmission Coefficient)
\item[$E$] 能量
\item[$m$] 质量
\item[$c$] 光速
\item[$P$] 概率
\item[$T$] 时间
\item[$v$] 速度
\item[劝  学] 君子曰:学不可以已。青,取之于蓝,而青于蓝;冰,水为之,而寒于水。
  木直中绳。(车柔)以为轮,其曲中规。虽有槁暴,不复挺者,(车柔)使之然也。故木
  受绳则直, 金就砺则利,君子博学而日参省乎己,则知明而行无过矣。吾尝终日而思
  矣,  不如须臾之所学也;吾尝(足齐)而望矣,不如登高之博见也。登高而招,臂非加
  长也,  而见者远;  顺风而呼,  声非加疾也,而闻者彰。假舆马者,非利足也,而致
  千里;假舟楫者,非能水也,而绝江河,  君子生非异也,善假于物也。积土成山,风雨
  兴焉;积水成渊,蛟龙生焉;积善成德,而神明自得,圣心备焉。故不积跬步,无以至千
  里;不积小流,无以成江海。骐骥一跃,不能十步;驽马十驾,功在不舍。锲而舍之,朽
  木不折;  锲而不舍,金石可镂。蚓无爪牙之利,筋骨之强,上食埃土,下饮黄泉,用心
  一也。蟹六跪而二螯,非蛇鳝之穴无可寄托者,用心躁也。\pozhehao{} 荀况
\end{denotation}
