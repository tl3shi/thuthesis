\begin{resume}

  \resumeitem{个人简历}

  1989 年 12 月 30 日出生于 重庆市石柱土家族自治县。
  
  2008 年 9 月考入 中南 大学 软件学院 软件工程专业,2012 年 7 月本科毕业并获得工学学士学位。
  
  2012 年 9 月免试进入清华大学软件学院攻读工学硕士学位至今。

  \resumeitem{发表的学术论文} % 发表的和录用的合在一起

  \begin{enumerate}[{[}1{]}]
    \item \textbf{唐磊}, 李春平, 杨柳. 统计策略序列模式挖掘及其在软件缺陷预测中的应用. 计算机科学, 2013, 40(5): 164-167.
    \item Shi KanLe, Yong JunHai, \textbf{Tang Lei}, et al. Polar NURBS surface with curvature continuity, Computer Graphics Forum. 2013, 32(7): 363-370.
    \item \textbf{唐磊}, 施侃乐, 雍俊海等. 模型适应的凸包围多面体并行生成算法. 中国科学:信息科学, 2014, 44(12): 1515-1526.
    \item 林建立, \textbf{唐磊}, 雍俊海等.多边形网格的非流形封闭三角形网格正则化. 计算机辅助设计与图形学学报,2014,26(10):1557-1566.
  \end{enumerate}

%  \resumeitem{研究成果} % 有就写,没有就删除
%  \begin{enumerate}[{[}1{]}]
%  \item 任天令, 杨轶, 朱一平, 等. 硅基铁电微声学传感器畴极化区域控制和电极连接的
%    方法: 中国, CN1602118A. (中国专利公开号.)
%  \item Ren T L, Yang Y, Zhu Y P, et al. Piezoelectric micro acoustic sensor
%    based on ferroelectric materials: USA, No.11/215, 102. (美国发明专利申请号.)
%  \end{enumerate}

\end{resume}
